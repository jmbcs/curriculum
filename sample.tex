%%%%%%%%%%%%%%%%%
% This is an sample CV template created using altacv.cls
% (v1.3, 10 May 2020) written by LianTze Lim (liantze@gmail.com). Now compiles with pdfLaTeX, XeLaTeX and LuaLaTeX.
%
%% It may be distributed and/or modified under the
%% conditions of the LaTeX Project Public License, either version 1.3
%% of this license or (at your option) any later version.
%% The latest version of this license is in
%%    http://www.latex-project.org/lppl.txt
%% and version 1.3 or later is part of all distributions of LaTeX
%% version 2003/12/01 or later.
%%%%%%%%%%%%%%%%

%% If you are using \orcid or academicons
%% icons, make sure you have the academicons
%% option here, and compile with XeLaTeX
%% or LuaLaTeX.
% \documentclass[10pt,a4paper,academicons]{altacv}

%% Use the "normalphoto" option if you want a normal photo instead of cropped to a circle
% \documentclass[10pt,a4paper,normalphoto]{altacv}

\documentclass[10pt,a4paper,ragged2e,withhyper]{altacv}
%% AltaCV uses the fontawesome5 and academicons fonts
%% and packages.
%% See http://texdoc.net/pkg/fontawesome5 and http://texdoc.net/pkg/academicons for full list of symbols. You MUST compile with XeLaTeX or LuaLaTeX if you want to use academicons.

% Change the page layout if you need to
\geometry{left=1.25cm,right=1.25cm,top=1.5cm,bottom=1.5cm,columnsep=1.2cm}

% The paracol package lets you typeset columns of text in parallel
\usepackage{paracol}
\usepackage{hyperref}
% Change the font if you want to, depending on whether
% you're using pdflatex or xelatex/lualatex
\ifxetexorluatex
  % If using xelatex or lualatex:
  \setmainfont{Roboto Slab}
  \setsansfont{Lato}
  \renewcommand{\familydefault}{\sfdefault}
\else
  % If using pdflatex:
  \usepackage[rm]{roboto}
  \usepackage[defaultsans]{lato}
  % \usepackage{sourcesanspro}
  \renewcommand{\familydefault}{\sfdefault}
\fi

% Change the colours if you want to

\definecolor{DarkPastelRed}{HTML}{450808}
\definecolor{PastelRed}{HTML}{8F0D0D}
\definecolor{GoldenEarth}{HTML}{E7D192}
\definecolor{bronze}{rgb}{0.8, 0.5, 0.2}
\definecolor{burgundy}{rgb}{0.45, 0.15, 0.15}

\definecolor{gray}{rgb}{0.25, 0.25, 0.25}
\definecolor{darkgray}{rgb}{0.152, 0.125, 0.125}
\definecolor{megadarkgray}{rgb}{0.1, 0.1, 0.1}

\colorlet{name}{megadarkgray}
\colorlet{tagline}{PastelRed}
\colorlet{heading}{burgundy}
\colorlet{headingrule}{bronze}
\colorlet{subheading}{PastelRed}
\colorlet{accent}{PastelRed}
\colorlet{emphasis}{darkgray}
\colorlet{body}{gray}

% Change some fonts, if necessary
\renewcommand{\namefont}{\huge\rmfamily\bfseries}
\renewcommand{\personalinfofont}{\footnotesize}
\renewcommand{\cvsectionfont}{\Large\rmfamily\bfseries}
\renewcommand{\cvsubsectionfont}{\large\bfseries}


% Change the bullets for itemize and rating marker
% for \cvskill if you want to
\renewcommand{\itemmarker}{{\normalsize\textbullet}}
\renewcommand{\ratingmarker}{\faCircle}

%% sample.bib contains your publications
\addbibresource{sample.bib}

\begin{document}
\name{Júlio Miguel Braz da Costa Silva}
\tagline{M.Sc Electronics and Telecommunications Engineering}
%% You can add multiple photos on the left or right
\photoR{3.5cm}{foto_perfil.jpg}
% \photoL{2.5cm}{Yacht_High,Suitcase_High}

\personalinfo{%
  % Not all of these are required!
 
  
  \medskip

  \phone{(+351) 924 002 628}

  \smallskip

  \email{julio.m.b.c.silva@gmail.com}
  \linkedin{julio-miguel-silva}

  \smallskip

  \location{Aveiro, Portugal}
  
  %% You MUST add the academicons option to \documentclass, then compile with LuaLaTeX or XeLaTeX, if you want to use \orcid or other academicons commands.
  % \orcid{0000-0000-0000-0000}
  %% You can add your own arbtrary detail with
  %% \printinfo{symbol}{detail}[optional hyperlink prefix]
  % \printinfo{\faPaw}{Hey ho!}[https://example.com/]
  %% Or you can declare your own field with
  %% \NewInfoFiled{fieldname}{symbol}[optional hyperlink prefix] and use it:
  % \NewInfoField{gitlab}{\faGitlab}[https://gitlab.com/]
  % \gitlab{your_id}
}

\makecvheader
%% Depending on your tastes, you may want to make fonts of itemize environments slightly smaller
% \AtBeginEnvironment{itemize}{\small}

%% Set the left/right column width ratio to 6:4.
\columnratio{0.62}

% Start a 2-column paracol. Both the left and right columns will automatically
% break across pages if things get too long.

\begin{paracol}{2}
  \cvsection{Work Experience}
  
  \cvevent{R\&D Engineer - Data Analysis - (Full/Hybrid) Remote}{Wavecom}{Jan 2021 – Present Time}{Aveiro, Portugal}

  \begin{itemize}
  \item[] \underline {\textbf{BigHPC}} \url{ https://bighpc.wavecom.pt/}
  \smallskip
  \begin{itemize}
      \item Research and Development of a monitoring solution for High Performance Computers. (Full pipeline of collecting metrics, data processing, data clean,data storage, data retention and data visualization)
      \item Moderator of the  Webinar "Is HPC ready for “Big” Data Storage?"
      \item Work with partners from different different nationalities.
  
    \end{itemize}

  \item[] \underline {\textbf{Public Network Monitoring}}
    \smallskip
      \begin{itemize}
      \item Research and Development of an architecture capable of monitoring networks.
      \item Creation of dashboarding visualizations for clients that required both the status and the history of their network. 
      \end{itemize}
    
\end{itemize}


  \cvsection{Education}

  \cvevent{Master Degree in Electronics and Telecommunications Engineering}{University of Aveiro}{2014 – 2020}{Aveiro, Portugal}
  \divider

\cvevent{Master Thesis -  19/20.}{University of Aveiro}{2019-2020 (1 year)}{Aveiro - Portugal} 
\begin{itemize}

\item[] \underline {\textbf{Recovery and Identification of Moments in Images}}
\item[] 
  \begin{itemize}
    \item Development of a python automatic image retrieval system capable of associating images to moments described in text.
 
    
  \end{itemize}

\end{itemize}

\divider

\cvevent{Computer Vision Project}{University of Aveiro}{2019 (3 months)}{Aveiro - Portugal}

\begin{itemize}
  \item[] \underline {\textbf{Automatic License Plate Detector}}
  \item[] 
    \begin{itemize}
      \item Development of a prototype system capable of validating pictures of license plates in real time.
      
    \end{itemize}
  \end{itemize}




% \cvsection{A Day of My Life}

% % Adapted from @Jake's answer from http://tex.stackexchange.com/a/82729/226
% % \wheelchart{outer radius}{inner radius}{
% % comma-separated list of value/text width/color/detail}
% \wheelchart{1.5cm}{0.5cm}{%
%   6/8em/accent!30/{Sleep,\\beautiful sleep},
%   3/8em/accent!40/Hopeful novelist by night,
%   8/8em/accent!60/Daytime job,
%   2/10em/accent/Sports and relaxation,
%   5/6em/accent!20/Spending time with family
% }

% use ONLY \newpage if you want to force a page break for
% ONLY the current column


\cvsection{Publications}

\nocite{*}

\renewbibmacro{in:}{}

\printbibliography[heading=pubtype,title={\printinfo{\faFile*[regular]}{Article}},type=article]


% \cvsection{Languages}

% \cvskill{Portuguese (Native)}{5}
% \divider

% \cvskill{English (C1)}{4}
% \divider


%% Switch to the right column. This will now automatically move to the second
%% page if the content is too long.
\switchcolumn




\cvsection{Tech Skills}
\cvtag{Python}
\cvtag{Docker Compose}
\cvtag{PromQL}
\cvtag{SQL}
\cvtag{TimescaleDB}  
\cvtag{Docker}
\cvtag{Linux}
\cvtag{Elasticsearch}
\cvtag{Kibana}
\cvtag{Grafana}
\cvtag{Postgres}
\cvtag{Prometheus}
\cvtag{Promscale}
\cvtag{TimescaleDB}
\cvtag{Fluentbit}
\cvtag{HPC}
\cvtag{JSON}
\cvtag{MATLAB}
\cvtag{Singularity}
\cvtag{Gitlab}
\cvtag{bash}
\cvtag{vmagent}
\cvtag{vmalert}
\cvtag{victoriametrics}


\cvsection{Python libraries}
\cvtag{Seaborn}
\cvtag{Pandas}
\cvtag{PandasSQL}
\cvtag{NumPy}
\cvtag{Psutil}
\cvtag{TensorFlow}
\cvtag{Keras}
\cvtag{ImageAI}
\cvtag{Matplotlib}
\cvtag{OpenCV}
\cvtag{SpaCy}
\cvtag{SQLalchemy}




\cvsection{Knowledge / Interests}{}{}{}
\cvtag{Metrics Extraction/Processing}
\cvtag{Databases}
\cvtag{Pattern Detection}
\cvtag{Data Science}\\
\cvtag{Automatization of tasks}
\cvtag{Data Visualization}
\cvtag{Data Processing}
\cvtag{Computer Vision}

\cvtag{Scene Recognition}
\cvtag{Image Processing}\\
\cvtag{Image Classification}
\cvtag{Deep Learning}\\
\cvtag{Neural Networks}
\cvtag{Feature Extraction}
\cvtag{Natural Language Processing}\\
\cvtag{Machine Learning}
\cvtag{Object Detection}\\






\cvsection{Strengths}
\cvtag{Self-Taught}
\cvtag{Communication}\\
\cvtag{Analytical Thinking}
\cvtag{Research}
\cvtag{Teamwork}\\
\cvtag{Attention to detail}
\cvtag{Hard-Working}\\
\cvtag{Eager to learn}
\cvtag{Innovative Thinking}
\cvtag{Easy to work with}
\cvtag{Patience}

\cvsection{Languages}

\cvtag{Portuguese (Native)}
\cvtag{English (Fluent)}


\cvsection{Driver Licence}

\cvtag{B1}
\cvtag{B}





%% Yeah I didn't spend too much time making all the
%% spacing consistent... sorry. Use \smallskip, \medskip,
%% \bigskip, \vpsace etc to make ajustments.

\end{paracol}


\end{document}
